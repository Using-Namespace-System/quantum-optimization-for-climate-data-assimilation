\documentclass{article}
\usepackage{amsmath}
\usepackage{amsfonts}
\usepackage{graphicx}

\title{Diophantine Quantum Subspace Expansion for 4D-Var}
\author{}
\date{}

\begin{document}
\maketitle

\section{Introduction}

The Four-Dimensional Variational (4D-Var) data assimilation problem can be framed as minimizing a cost function given by:

\begin{equation}
J(x) = \frac{1}{2} \| Hx - d \|^2 + \frac{1}{2} \| B^{-1}(x - x_0) \|^2,
\end{equation}

where \( H \) is the observation operator, \( d \) is the observed data vector, \( B \) is the background error covariance matrix, and \( x_0 \) is the prior estimate. 

We can express this in a linear form:

\begin{equation}
Ax = b,
\end{equation}

where

\[
A = H^TH + B^{-1}, \quad b = H^Td + B^{-1}x_0.
\]

We will explore the Diophantine formulation of the quantum subspace expansion method to solve this system.

\section{Diophantine System Formulation}

A Diophantine system involves finding integer solutions \( x \in \mathbb{Z}^n \) such that:

\begin{equation}
A \mathbf{z} = \mathbf{b}, \quad \text{with } \mathbf{z} \in \mathbb{Z}^n.
\end{equation}

In the context of 4D-Var, the goal is to find integer solutions \( \mathbf{z} \) that minimize the cost function \( J(x) \).

### Step 1: Quantum Subspace Definition

Define a subspace \( S \) spanned by a set of basis vectors \( \{|v_1\rangle, |v_2\rangle, \ldots, |v_k\rangle\} \) where each basis vector corresponds to a candidate solution in \( \mathbb{C}^n \). The expansion within this subspace can be represented as:

\[
|x\rangle = \sum_{i=1}^{k} c_i |v_i\rangle,
\]

where \( c_i \) are complex coefficients to be determined.

### Step 2: Projection onto the Subspace

We can project the observed data \( |b\rangle \) onto the subspace \( S \):

\begin{equation}
|b\rangle \approx \sum_{i=1}^{k} \alpha_i |v_i\rangle,
\end{equation}

where \( \alpha_i \) are coefficients representing the projection.

### Step 3: Formulation of Integer Constraints

The integer solutions can be obtained by framing the projection as a Diophantine equation. We reformulate the problem as:

\begin{equation}
A \mathbf{z} \approx \mathbf{b},
\end{equation}

leading to:

\[
\sum_{j=1}^{m} a_{ij} z_j = b_i, \quad \forall i,
\]

where \( a_{ij} \in \mathbb{Z} \).

### Step 4: Quantum Implementation of the Subspace Expansion

1. **State Preparation:** Prepare the quantum state \( |b\rangle \) representing the observed data.

2. **Subspace Encoding:** Encode the subspace \( S \) into the quantum state using a quantum circuit that constructs the basis states \( |v_i\rangle \).

3. **Optimization Procedure:** Use a variational approach to optimize the coefficients \( c_i \):

\begin{equation}
\min_{\mathbf{c}} \left\| A\left(\sum_{i=1}^{k} c_i |v_i\rangle\right) - |b\rangle \right\|^2.
\end{equation}

4. **Quantum Measurements:** Measure the state to extract the coefficients \( c_i \) that correspond to the integer solutions.

### Step 5: Solving the Diophantine System

The optimization leads to finding \( \mathbf{z} \):

\begin{equation}
\mathbf{z} = \left( A^T A \right)^{-1} A^T \mathbf{b},
\end{equation}

which may yield non-integer solutions. To ensure integer constraints, one can apply rounding or integer programming techniques to adjust \( \mathbf{z} \) to the nearest integers that satisfy the original equations.

### Step 6: Validating the Solution

To validate the integer solution, we check:

\begin{equation}
A \mathbf{z}_{\text{int}} = \mathbf{b}_{\text{int}},
\end{equation}

where \( \mathbf{z}_{\text{int}} \) represents the adjusted integer solution and \( \mathbf{b}_{\text{int}} \) is the corresponding integer observation.

### Conclusion

This Diophantine approach utilizing quantum subspace expansion provides a framework for finding integer solutions to the linear system arising from the 4D-Var problem. By projecting onto a suitable subspace and employing quantum techniques, we can effectively solve the data assimilation problem while ensuring integer constraints.

\end{document}
