\documentclass{article}
\usepackage{amsmath}
\usepackage{amsfonts}
\usepackage{graphicx}
\usepackage{caption}
\usepackage{subcaption}

\title{Detailed Plan for Laser Implementation in Analog Quantum Systems}
\author{}
\date{}

\begin{document}

\maketitle

\section{Introduction}
This document outlines a detailed plan for implementing an analog quantum system using lasers. The focus will be on the use of coherent light sources to simulate quantum states and dynamics, specifically for modeling the discretized Schrödinger equation.

\section{System Design}

\subsection{Architecture}

\begin{itemize}
    \item \textbf{Laser Source:} A coherent light source (e.g., a diode laser) emitting light at a specific wavelength.
    \item \textbf{Optical Components:} Beam splitters, mirrors, lenses, and phase shifters for manipulating light.
    \item \textbf{Detector:} Photodetector or CCD camera for measuring the output light intensity and phase.
\end{itemize}

\subsection{Setup Configuration}

\begin{figure}[h]
    \centering
    \includegraphics[width=0.7\textwidth]{laser_setup.png} % Placeholder for a diagram
    \caption{Schematic of the laser-based analog quantum system.}
    \label{fig:laser_setup}
\end{figure}

\subsection{Key Parameters}
\begin{itemize}
    \item \textbf{Wavelength (\(\lambda\)):} The wavelength of the laser light, typically in the range of 400-700 nm.
    \item \textbf{Beam Diameter (\(d\)):} The diameter of the laser beam, affecting the intensity distribution.
    \item \textbf{Intensity (\(I\)):} The power per unit area, calculated as:
    \[
    I = \frac{P}{\pi \left(\frac{d}{2}\right)^2}
    \]
    where \(P\) is the laser power.
\end{itemize}

\section{Operational Steps}

\subsection{Initialization}
1. **Laser Activation:** Turn on the diode laser and allow it to stabilize.
2. **Alignment:** Use mirrors and lenses to align the laser beam through the optical components.

\subsection{State Preparation}
1. **Creating Superposition States:**
   \[
   |\psi\rangle = \alpha |0\rangle + \beta |1\rangle
   \]
   where \(|\alpha|^2 + |\beta|^2 = 1\).
2. **Interference Setup:** Use beam splitters to create interference patterns representing quantum superpositions.

\subsection{Hamiltonian Simulation}
The system can be modeled using an effective Hamiltonian:
\[
H = \frac{\hat{p}^2}{2m} + V(\hat{x})
\]
where:
- \(\hat{p}\) is the momentum operator,
- \(m\) is the mass of the particle,
- \(V(\hat{x})\) is the potential energy, which can be simulated using the phase modulation of the laser.

\subsection{Measurement}
1. **Intensity Measurement:** Use a photodetector to measure the intensity of the output light:
   \[
   I_{\text{out}} = k \cdot I_{\text{in}} \cdot \left( \sin^2(\theta) \right)
   \]
   where \(k\) is a calibration constant and \(\theta\) is the phase shift introduced by the optical components.
2. **Data Analysis:** Analyze the measurement data to extract information about the quantum states and energy eigenvalues.

\section{Mathematical Model}

\subsection{Coherent States}
The coherent state of the laser can be expressed as:
\[
|\alpha\rangle = e^{-|\alpha|^2/2} \sum_{n=0}^{\infty} \frac{\alpha^n}{\sqrt{n!}} |n\rangle
\]
where \(|\alpha|^2\) represents the average photon number.

\subsection{Propagation of Light}
The propagation of light through a medium can be described using the wave equation:
\[
\nabla^2 E - \frac{1}{c^2} \frac{\partial^2 E}{\partial t^2} = 0
\]
where \(E\) is the electric field of the light wave, and \(c\) is the speed of light.

\subsection{Interference Patterns}
The intensity of the combined light from two coherent sources can be expressed as:
\[
I = I_1 + I_2 + 2\sqrt{I_1 I_2} \cos(\Delta \phi)
\]
where \(\Delta \phi\) is the phase difference between the two sources.

\section{Conclusion}
This document presents a comprehensive plan for implementing a laser-based analog quantum system. By leveraging basic optical components and coherent light sources, it is possible to simulate quantum states and dynamics effectively. The mathematical foundations outlined here provide the necessary tools for analyzing and interpreting the results obtained from the system.

\end{document}
