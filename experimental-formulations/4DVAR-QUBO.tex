\documentclass{article}
\usepackage{graphicx} % Required for inserting images
\usepackage{amsmath}
\usepackage{amsfonts}
\usepackage{listings}
\usepackage{xcolor}

\lstset{
    backgroundcolor=\color{lightgray},
    basicstyle=\ttfamily,
    breaklines=true
}

\title{quantum optimization for climate data assimilation}
\author{Brian Recktenwall-Calvet}
\date{October 2024}
\begin{document}
\maketitle
\section{Quadratic Unconstrained Optimization Problem Formulation}

\subsection{Define the Problem Components}

We start with the following components in our 4D-Var framework:

\begin{itemize}
    \item Background State: \( \mathbf{x}_b \)
    \item Initial Guess for Analysis Increment: \( \mathbf{dx}^0 \)
    \item Observations: \( \mathbf{y}_{obs} \)
    \item Error Covariances: \( B_{inv} \) (background error) and \( R_{inv} \) (observation error)
    \item Nonlinear Operator: \( \mathcal{N}(\mathbf{x}) \)
\end{itemize}

\subsection{Linearization of the Nonlinear Operator}

Using the tangent linear approximation around the trajectory from the first guess, we have:

\[
\mathbf{y} = \mathcal{N}(\mathbf{x}_b + \mathbf{dx}) \approx \mathcal{N}(\mathbf{x}_b + \mathbf{dx}^0) + \mathcal{N}'(\mathbf{x}^0) \cdot \mathbf{dx}
\]

Denote:

\[
\mathbf{y}^0 = \mathcal{N}(\mathbf{x}_b + \mathbf{dx}^0)
\]

Thus, we can write:

\[
\mathbf{y} \approx \mathbf{y}^0 + \mathcal{N}'(\mathbf{x}^0) \cdot \mathbf{dx}
\]

\subsection{Formulate the Cost Function}

The cost function \( J(\mathbf{dx}) \) is defined as:

\[
J(\mathbf{dx}) = \frac{1}{2} \|\mathbf{dx}\|^2_{B_{inv}} + \frac{1}{2} \|\mathbf{y}_{obs} - \mathbf{y}^0 - \mathcal{N}'(\mathbf{x}^0) \cdot \mathbf{dx}\|^2_{R_{inv}}
\]

Expanding the observation term gives:

\[
\|\mathbf{y}_{obs} - \mathbf{y}^0 - \mathcal{N}'(\mathbf{x}^0) \cdot \mathbf{dx}\|^2_{R_{inv}} = (\mathbf{y}_{obs} - \mathbf{y}^0 - \mathcal{N}'(\mathbf{x}^0) \cdot \mathbf{dx})^T R_{inv} (\mathbf{y}_{obs} - \mathbf{y}^0 - \mathcal{N}'(\mathbf{x}^0) \cdot \mathbf{dx})
\]

\subsection{Constructing the QUBO Formulation}

Combining the components, we rewrite the cost function \( J(\mathbf{dx}) \):

\[
J(\mathbf{dx}) = \frac{1}{2} \mathbf{dx}^T B_{inv} \mathbf{dx} + \frac{1}{2} (\mathbf{y}_{obs} - \mathbf{y}^0 - \mathcal{N}'(\mathbf{x}^0) \cdot \mathbf{dx})^T R_{inv} (\mathbf{y}_{obs} - \mathbf{y}^0 - \mathcal{N}'(\mathbf{x}^0) \cdot \mathbf{dx})
\]

This expands to:


\begin{align*}
J(\mathbf{dx}) &= \frac{1}{2} \mathbf{dx}^T B_{inv} \mathbf{dx} \\
&\quad + \frac{1}{2} (\mathbf{y}_{obs} - \mathbf{y}^0)^T R_{inv} (\mathbf{y}_{obs} - \mathbf{y}^0) \\
&\quad - \mathbf{dx}^T \mathcal{N}'(\mathbf{x}^0)^T R_{inv} (\mathbf{y}_{obs} - \mathbf{y}^0) \\
&\quad + \frac{1}{2} \mathbf{dx}^T \mathcal{N}'(\mathbf{x}^0)^T R_{inv} \mathcal{N}'(\mathbf{x}^0) \mathbf{dx}
\end{align*}

Thus, we can express:

\[
J(\mathbf{dx}) = \frac{1}{2} \mathbf{dx}^T \left( B_{inv} + \mathcal{N}'(\mathbf{x}^0)^T R_{inv} \mathcal{N}'(\mathbf{x}^0) \right) \mathbf{dx} - \mathbf{dx}^T \mathcal{N}'(\mathbf{x}^0)^T R_{inv} (\mathbf{y}_{obs} - \mathbf{y}^0) + C
\]

where \( C \) is a constant term independent of \( \mathbf{dx} \).

\subsection{Define the QUBO Problem}

To convert this into QUBO format, we need to express it as:

\[
\text{Minimize } J(\mathbf{b}) = \frac{1}{2} \mathbf{b}^T Q \mathbf{b} + \mathbf{c}^T \mathbf{b} + C
\]

\subsubsection{Binary Mapping}

Define a binary representation of the analysis increment \( \mathbf{dx} \):

\begin{itemize}
    \item Convert each \( dx_i \) to binary variables \( b_i \).
\end{itemize}

After substituting \( \mathbf{dx} = f(\mathbf{b}) \) into the cost function and expressing it as a quadratic form with respect to the binary variables, the final QUBO representation becomes:

\[
\text{Minimize } J(\mathbf{b}) = \frac{1}{2} \mathbf{b}^T Q \mathbf{b} + \mathbf{c}^T \mathbf{b} + C
\]

where \( Q \) and \( \mathbf{c} \) are derived from the previous expansions.

\end{document}
