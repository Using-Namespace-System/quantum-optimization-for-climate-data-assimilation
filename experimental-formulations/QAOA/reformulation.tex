\documentclass{article}
\usepackage{graphicx} % Required for inserting images
\usepackage{amsmath}
\usepackage{amsfonts}
\usepackage{listings}
\usepackage{xcolor}

\lstset{
    backgroundcolor=\color{lightgray},
    basicstyle=\ttfamily,
    breaklines=true
}

\title{quantum optimization for climate data assimilation}
\author{Brian Recktenwall-Calvet}
\date{October 2024}
\begin{document}
\maketitle
\section{4DVAR Cost Function}

The 4DVAR cost function is defined as:

\begin{equation}
J(\delta x_0) = \delta x_0^T Q_0^{-1} \delta x_0 + d_{1:L}^T R_{1:L}^{-1} d_{1:L}
\end{equation}

where:
\begin{itemize}
    \item \( \delta x_0 = x_0 - x^f_0 \) is the analysis increment,
    \item \( Q_0 \) is the background error covariance matrix,
    \item \( d_{1:L} \) is the observation departure,
    \item \( R_{1:L} \) is the observation error covariance matrix.
\end{itemize}

\section{Observation Departure}

The observation departure \( d_k \) for each time step \( k \) is given by:

\begin{equation}
d_k = y^o_k - H M_{k|0}(x^f_0 + \delta x_0)
\end{equation}

where:
\begin{itemize}
    \item \( y^o_k \) is the observed value at time \( k \),
    \item \( H \) is the linear observation operator,
    \item \( M_{k|0} \) is the nonlinear model forecast from time \( t=0 \) to time \( k \).
\end{itemize}

\section{Full Observation Departure Vector}

The full observation departure vector is defined as:

\begin{equation}
d_{1:L}^T = \begin{bmatrix}
d_1^T \\
\vdots \\
d_L^T
\end{bmatrix}
\end{equation}

Substituting the individual departures into the cost function gives:

\begin{equation}
J(\delta x_0) = \delta x_0^T Q_0^{-1} \delta x_0 + \left( \begin{bmatrix}
y^o_1 - H M_{1|0}(x^f_0 + \delta x_0) \\
\vdots \\
y^o_L - H M_{L|0}(x^f_0 + \delta x_0)
\end{bmatrix} \right)^T R_{1:L}^{-1} \left( \begin{bmatrix}
y^o_1 - H M_{1|0}(x^f_0 + \delta x_0) \\
\vdots \\
y^o_L - H M_{L|0}(x^f_0 + \delta x_0)
\end{bmatrix} \right)
\end{equation}

\section{Deriving the Gradient}

To minimize \( J(\delta x_0) \), we need the gradient:

\begin{equation}
\frac{\partial J(\delta x_0)}{\partial \delta x_0} = 2 Q_0^{-1} \delta x_0 - 2 M^T_{1:L} H^T R_{1:L}^{-1} d_{1:L}
\end{equation}

\subsection*{Proof of Gradient Calculation}

1. **First Term**: The first term \( \delta x_0^T Q_0^{-1} \delta x_0 \) is a quadratic form.
   - Its gradient with respect to \( \delta x_0 \) is \( 2 Q_0^{-1} \delta x_0 \).

2. **Second Term**: For the second term \( d_{1:L}^T R_{1:L}^{-1} d_{1:L} \):
   - The gradient can be computed by the chain rule, recognizing that \( d_{1:L} \) depends on \( \delta x_0 \).
   - The result is \( -2 M^T_{1:L} H^T R_{1:L}^{-1} d_{1:L} \).

Combining these gives:

\begin{equation}
\frac{\partial J(\delta x_0)}{\partial \delta x_0} = 2 Q_0^{-1} \delta x_0 - 2 M^T_{1:L} H^T R_{1:L}^{-1} d_{1:L}
\end{equation}

\section{Relating to QAOA}

Now, we can relate the optimization of the 4DVAR cost function to the Quantum Approximate Optimization Algorithm (QAOA).

1. **Encoding into a Hamiltonian**:
   - The cost function \( J(\delta x_0) \) can be expressed in a quadratic form suitable for QAOA.
   - Define the cost Hamiltonian \( H_C \) based on the 4DVAR problem.

2. **Parameterization**:
   - QAOA uses a parameterized quantum circuit to explore the solution space, adjusting parameters to minimize \( H_C \).

3. **Optimization**:
   - Classical optimization techniques (like gradient descent) can be used to find the optimal parameters that minimize the cost function.
   - This is analogous to optimizing \( \delta x_0 \) in 4DVAR.

\section{Introduction to QAOA}

The Quantum Approximate Optimization Algorithm (QAOA) is designed to solve combinatorial optimization problems on quantum computers. It utilizes a variational approach, where a parameterized quantum circuit is iteratively optimized to minimize a cost function.

\section{QAOA Framework}

The QAOA consists of the following main components:

1. **Cost Function**: We need to express our optimization problem in terms of a cost Hamiltonian \( H_C \). This Hamiltonian should reflect the structure of the optimization problem.

2. **Parameterization**: QAOA introduces two sets of parameters:
   \[
   \gamma = (\gamma_1, \gamma_2, \ldots, \gamma_p)
   \]
   \[
   \beta = (\beta_1, \beta_2, \ldots, \beta_p)
   \]
   These parameters control the evolution of the quantum state through \( p \) layers of the circuit.

3. **Quantum Circuit**: The QAOA circuit is built as follows:
   \[
   |\psi(\gamma, \beta)\rangle = e^{-i \beta_p H_B} e^{-i \gamma_p H_C} \cdots e^{-i \beta_1 H_B} e^{-i \gamma_1 H_C} |s\rangle
   \]
   where \( |s\rangle \) is some initial state, usually chosen as \( |0\rangle^{\otimes n} \).

\section{Defining the Cost Hamiltonian}

To apply QAOA to our 4DVAR cost function \( J(\delta x_0) \), we need to define a cost Hamiltonian \( H_C \) based on the quadratic form of \( J(\delta x_0) \).

Given:
\[
J(\delta x_0) = \delta x_0^T Q_0^{-1} \delta x_0 + d_{1:L}^T R_{1:L}^{-1} d_{1:L}
\]

We can express this in terms of binary variables, which can be encoded into qubits.

1. **Mapping to Binary Variables**: We use a mapping matrix \( G \) to relate the continuous variable \( \delta x_0 \) to binary variables \( b \):
   \[
   \delta x_0 \approx \frac{1}{\alpha} G b
   \]

2. **Cost Hamiltonian**:
   The cost Hamiltonian is then defined as:
   \[
   H_C = \frac{1}{\alpha^2} G^T \left( Q_0^{-1} + \tilde{M}^T_{1:L|0} H^T_{1:L} R_{1:L}^{-1} H_{1:L} \tilde{M}_{1:L|0} \right) G
   \]

This formulation allows us to express the minimization of the cost function \( J(\delta x_0) \) in terms of the Hamiltonian \( H_C \).

\section{QAOA Implementation}

1. **Prepare Initial State**: We start with an initial quantum state, often taken as \( |0\rangle^{\otimes N} \) where \( N \) is the number of qubits.

2. **Layered Quantum Circuit**:
   The quantum circuit for QAOA consists of \( p \) layers of gates. Each layer consists of:
   - A \( R_z \) rotation for the cost Hamiltonian.
   - A \( R_y \) rotation for the mixer Hamiltonian \( H_B \), typically chosen as:
   \[
   H_B = \sum_{i} X_i
   \]

3. **Quantum State Evolution**:
   The state evolves through the parameterized circuit as:
   \[
   |\psi(\gamma, \beta)\rangle = e^{-i \beta_p H_B} e^{-i \gamma_p H_C} \cdots e^{-i \beta_1 H_B} e^{-i \gamma_1 H_C} |0\rangle^{\otimes N}
   \]

\section{Measurement and Classical Optimization}

1. **Measurement**: After applying the quantum circuit, we measure the state to obtain samples. Each measurement corresponds to a possible solution of the original optimization problem.

2. **Objective Function Evaluation**: For each sampled state \( b \), we evaluate the cost function \( J(b) \).

3. **Classical Optimization**:
   - We use a classical optimization method to adjust the parameters \( \gamma \) and \( \beta \) to minimize the expected value of the cost Hamiltonian:
   \[
   E[H_C] = \langle \psi(\gamma, \beta) | H_C | \psi(\gamma, \beta) \rangle
   \]

4. **Iterative Updates**: The parameters are iteratively updated based on the results of the measurements until convergence is reached.

\section{Convergence and Output}

The goal is to find the optimal parameters \( \gamma^* \) and \( \beta^* \) that minimize the expected value of the cost Hamiltonian, leading to the optimal solution for \( \delta x_0 \) that approximates the minimum of the original 4DVAR cost function.

\section{Step 1: Define the 4DVAR Cost Function as a QUBO}

The original 4DVAR cost function is defined as:

\begin{equation}
J(\delta x_0) = \delta x_0^T Q_0^{-1} \delta x_0 + d_{1:L}^T R_{1:L}^{-1} d_{1:L},
\end{equation}

where \( \delta x_0 \) represents the analysis increment. We can reformulate this cost function in terms of binary variables \( b \) to approximate \( \delta x_0 \), leading to a quadratic form:

\begin{equation}
J(b) = b^T Q b,
\end{equation}

where \( Q \) incorporates the contributions from the covariance matrices \( Q_0 \) and \( R_{1:L} \).

\section{Step 2: Establish the Cost Hamiltonian for QAOA}

In QAOA, the optimization problem is expressed using a Hamiltonian \( H_C \):

\begin{equation}
H_C = \sum_{i,j} Q_{ij} b_i b_j,
\end{equation}

which represents the cost function we seek to minimize.

\section{Step 3: Describe the QAOA Circuit Structure}

The QAOA constructs a parameterized quantum circuit to encode the optimization problem:

\subsection{Initial State}
The process begins with a uniform superposition of all binary states, represented as:

\begin{equation}
|0\rangle^{\otimes N}.
\end{equation}

\subsection{Parameterized Circuit}
The quantum state evolves through \( p \) layers of gates, alternating between the cost Hamiltonian \( H_C \) and a mixing Hamiltonian \( H_B \):

\begin{equation}
|\psi(\gamma, \beta)\rangle = e^{-i \beta_p H_B} e^{-i \gamma_p H_C} \cdots e^{-i \beta_1 H_B} e^{-i \gamma_1 H_C} |0\rangle^{\otimes N}.
\end{equation}

Here, \( H_B \) typically facilitates transitions between binary states, allowing effective exploration of the solution space.

\section{Step 4: Demonstrate the Optimization Process}

\subsection{Measurement}
After applying the quantum circuit, we measure the resulting state to obtain binary outcomes \( b \).

\subsection{Objective Function Evaluation}
For each sampled state \( b \), the cost function is evaluated as:

\begin{equation}
J(b) = b^T Q b.
\end{equation}

\subsection{Classical Optimization}
The parameters \( \gamma \) and \( \beta \) are then adjusted using a classical optimization method to minimize the expected value of the Hamiltonian:

\begin{equation}
E[H_C] = \langle \psi(\gamma, \beta) | H_C | \psi(\gamma, \beta) \rangle.
\end{equation}

This process directly minimizes the QUBO formulation derived from the 4DVAR context.


\end{document}

