\documentclass{article}
\usepackage{amsmath}
\usepackage{amsfonts}
\usepackage{graphicx}

\begin{document}

\title{Unbalanced Penalization for the Knapsack Problem}
\author{}
\date{}
\maketitle

\section{Problem Definition}

The \textbf{Knapsack Problem} involves selecting a subset of items, each with a value \( v_i \) and weight \( w_i \), such that the total weight does not exceed a given limit \( W \). The goal is to maximize the total value:

\[
\text{Maximize} \quad \sum_{i=1}^{n} v_i x_i
\]

subject to:

\[
\sum_{i=1}^{n} w_i x_i \leq W
\]

where \( x_i \in \{0, 1\} \) indicates whether item \( i \) is included in the knapsack.

\section{Unbalanced Penalization Approach}

To incorporate penalties for violating the weight constraint, we modify the objective function as follows:

\subsection{Modified Objective Function}

The modified objective function is given by:

\[
\min_{x,s} \left( f(x) + p(x,s) \right)
\]

where:

\[
f(x) = -\sum_{i=1}^{n} v_i x_i
\]

The penalty function \( p(x,s) \) includes both a linear penalty for exceeding the weight constraint and a quadratic penalty:

\[
p(x,s) = -\lambda_1 \left( \sum_{i=1}^{n} w_i x_i - W \right) + \lambda_2 \left( \sum_{i=1}^{n} w_i x_i - W \right)^2
\]

Thus, the complete objective becomes:

\[
\min_{x} \left( -\sum_{i=1}^{n} v_i x_i - \lambda_1 \left( \sum_{i=1}^{n} w_i x_i - W \right) + \lambda_2 \left( \sum_{i=1}^{n} w_i x_i - W \right)^2 \right)
\]

\section{Parameter Definitions}

\begin{itemize}
    \item \( \lambda_1 \): Coefficient for the linear penalty when the weight exceeds \( W \).
    \item \( \lambda_2 \): Coefficient for the quadratic penalty, which imposes a larger penalty as the weight violation increases.
\end{itemize}

\section{Implementation Steps}

\subsection{Model Creation}

Using a mathematical programming library like Docplex, we define the Knapsack model:

1. \textbf{Binary Variable Definition:}

   Let \( x \) be a binary variable list representing item selection:

   \[
   x = \text{binary\_var\_list}(\text{range}(n), \text{name}="x")
   \]

2. \textbf{Objective Function Setup:}

   The objective is to minimize the negative total value:

   \[
   \text{cost} = -\sum_{i=1}^{n} x_i v_i
   \]

3. \textbf{Weight Constraint:}

   Add the constraint to ensure the total weight does not exceed \( W \):

   \[
   \sum_{i=1}^{n} w_i x_i \leq W
   \]

\section{Conversion to Ising Model}

To solve the problem using quantum algorithms, we need to convert the problem into an Ising model:

1. \textbf{Using Docplex to Ising Conversion:}

   The model is transformed into an Ising Hamiltonian, which typically has the form:

   \[
   H(z) = \sum_{i} h_i z_i + \sum_{i < j} J_{ij} z_i z_j + O
   \]

   Here, \( h_i \) are linear terms, \( J_{ij} \) are interaction terms, and \( O \) is a constant offset.

2. \textbf{Extracting Terms:}

   From the Ising model, we extract the linear and quadratic terms:

   - Single-qubit terms \( h_{\text{new}} \):
     \[
     h_{\text{new}} = \{(i,): \text{weight from Ising model}\}
     \]

   - Two-qubit interaction terms \( J_{\text{new}} \):
     \[
     J_{\text{new}} = \{(i,j): \text{interaction strength from Ising model}\}
     \]

\section{QAOA Circuit Implementation}

With the new Hamiltonian, the QAOA circuit is implemented:

1. \textbf{Circuit Definition:}

   The QAOA circuit alternates between applying the cost Hamiltonian and the mixer Hamiltonian. The circuit can be expressed mathematically as follows:

   \[
   U(H, \gamma) = e^{-i \gamma H_c}, \quad U(B, \beta) = e^{-i \beta X}
   \]

   where \( H_c \) is the cost Hamiltonian, and \( X \) is the mixer Hamiltonian.

2. \textbf{Sample Generation:}

   Execute the QAOA circuit to obtain samples:

   \[
   \text{samples\_unbalanced} = \text{samples\_dict}(qaoa\_circuit(gammas, betas, h_{\text{new}}, J_{\text{new}}, n))
   \]

3. \textbf{Valid Solution Filtering:}

   Retain samples that satisfy the weight constraint:

   \[
   \{ \text{valid samples} \, | \, \sum_{i=1}^{n} w_i x_i \leq W \}
   \]




\end{document}
