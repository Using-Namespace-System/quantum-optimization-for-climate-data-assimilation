\documentclass{article}
\usepackage{amsmath}
\usepackage{graphicx}

\begin{document}

\title{Use Case: Air Quality Monitoring}
\author{Brian Recktenwall-Calvet}
\date{October 2024}
\maketitle

\section{Objective}
To predict pollution levels based on air quality data collected from various sensors in an urban area.

\section{Data Collection}
\begin{itemize}
    \item \textbf{Variables}: Select relevant variables such as PM2.5, NOx, and Ozone levels.
    \item \textbf{Spatial Resolution}: Define a 3D grid with dimensions \(X\), \(Y\), and \(Z\) to represent the spatial distribution of pollution.
    \item \textbf{Temporal Resolution}: Gather hourly data over a 25-hour time window.
\end{itemize}

\textbf{Example Data Structure}:
\noindent
\begin{align*}
    \text{Number of variables} & : V = 4 \quad (\text{PM2.5, NOx, Ozone, etc}) \\
    \text{Grid Dimensions} & : X \times Y \times Z \quad (\text{e.g., } 10 \times 10 \times 5) \\
    \text{Number of time steps} & : T = 25 \\
    \text{Total data points} & : D = V \times (X \times Y \times Z) \times T.
\end{align*}

\subsection{Total Data Points}
The total number of data points can be calculated as:
\[
m = T \times \text{Total grid points} \times V = 25 \times 500 \times 4.
\]
Calculating this gives:
\[
m = 25 \times 500 \times 4 = 50000 \text{ data points}.
\]

\subsubsection{Number of Qubits Needed}
The number of qubits needed is:
\[
n = \lceil \log_2(m) \rceil = \lceil \log_2(50000) \rceil \approx 16.
\]

\subsection{Normalization}
Normalize the data to a suitable range (e.g., [0, 1]):
\[
    \text{NormalizedData}[v, x, y, z, t] = \frac{\text{Data}[v, x, y, z, t] - \text{min}_v}{\text{max}_v - \text{min}_v}.
\]

\subsection{Vectorization}
Flatten the normalized data into a vector:
\[
    \text{VectorizedData} = \text{reshape}(\text{NormalizedData}, [V \times (X \times Y \times Z) \times T, 1]).
\]

\subsection{Constructing the Cost Function Components}
\begin{itemize}
    \item \textbf{State Variables}:
    \begin{itemize}
        \item \(\delta x_0\): Analysis adjustment based on the analysis window.
        \item \(x^b\): Background state.
    \end{itemize}
    
    \item \textbf{Observation Variables}:
    \begin{itemize}
        \item \(y_k^o\): Observed data at time \(k\).
        \item \(H_k(x^b + \delta x_0)\): Model output at time \(k\).
    \end{itemize}
    
    \item \textbf{Covariance Matrices}:
    \begin{itemize}
        \item \(Q_0\): Background error covariance.
        \item \(R_k\): Observation error covariance.
    \end{itemize}
\end{itemize}

\section{Initial Forecast}
Generate the initial forecast based on the background state \(x^b\):
\[
    F_k = H_k(x^b) \quad \text{for } k = 1, 2, \ldots, T.
\]

\subsection{Calculate Residuals}
Compute the residuals between the initial forecasts and observations:
\[
    d_{1:L} = y_{1:L}^o - F_{1:L}.
\]

\section{Set Up Cost Function}
Formulate the cost function:
\[
    J(\delta x_0) = \delta x_0^T Q_0^{-1} \delta x_0 + d_{1:L}^T R_{1:L}^{-1} d_{1:L}.
\]

\subsection{Minimize Cost Function Using QAOA}

The optimization can be performed using the Quantum Approximate Optimization Algorithm (QAOA). Below are the detailed steps involved.

\subsection{Hamiltonian Construction}
Construct the Hamiltonian representing the cost function:
\[
    H = J(\delta x_0) = \delta x_0^T Q_0^{-1} \delta x_0 + \sum_{k=1}^{T} (y_k^o - H_k(x^b + \delta x_0))^T R_k^{-1} (y_k^o - H_k(x^b + \delta x_0)).
\]

\subsubsection{Parameter Initialization}
Choose initial parameters for the QAOA circuit:
\begin{itemize}
    \item \textbf{Number of Layers \(p\)}: Determine the depth of the QAOA circuit.
    \item \textbf{Parameters \(\gamma\) and \(\beta\)}: Initialize parameters for the unitary operators.
\end{itemize}

\subsubsection{Quantum Circuit Design}
Design the QAOA circuit:
\begin{itemize}
    \item \textbf{Apply Phase Separator}:
    \[
        U(C) = e^{-i \gamma H}
    \]

    \item \textbf{Apply Mixing Operator}:
    \[
        U(B) = e^{-i \beta B},
    \]
    where \(B\) is a mixing Hamiltonian chosen to promote exploration of the solution space.
\end{itemize}

\subsubsection{QAOA Circuit Construction}
Construct the QAOA circuit by alternating the application of the phase separator and the mixing operator:
\[
    U(\gamma, \beta) = (U(B) U(C))^p.
\]

\subsubsection{State Preparation}
Prepare the initial state of the quantum system:
\[
    | \psi_0 \rangle = |0\rangle^{\otimes n},
\]
where \(n\) is the number of qubits.

\subsubsection{Measurement}
After executing the QAOA circuit, measure the quantum state:
\begin{itemize}
    \item \textbf{Classical Post-Processing}: Analyze measurement outcomes to estimate \(\delta x_0\).
    \item \textbf{Cost Evaluation}: Evaluate the cost function \(J(\delta x_0)\) using classical outcomes.
\end{itemize}

\subsubsection{Parameter Optimization}
Use a classical optimization algorithm (e.g., gradient descent, COBYLA) to optimize parameters \(\gamma\) and \(\beta\):
\begin{align*}
    \text{Minimize } & J(\delta x_0) \\
    \text{with respect to } & \gamma, \beta.
\end{align*}

\section{Sensitivity Analysis or Linearization}
Instead of directly evaluating \(H_k\) for the updated state, use sensitivity analysis or a linear approximation based on the initial forecast:
\[
    F_k^a = F_k + H_k(\delta x_0).
\]
If \(H_k\) is nonlinear, linearize it around \(x^b\) to estimate the adjustment’s impact.

\end{document}
