\documentclass{article}
\usepackage{graphicx} % Required for inserting images
\usepackage{amsmath}
\usepackage{amsfonts}
\usepackage{listings}
\usepackage{xcolor}

\lstset{
    backgroundcolor=\color{lightgray},
    basicstyle=\ttfamily,
    breaklines=true
}

\title{quantum optimization for climate data assimilation}
\author{Brian Recktenwall-Calvet}
\date{October 2024}
\begin{document}

\maketitle

\section{Introduction}

4DVAR (Four-Dimensional Variational Data Assimilation), 


\section{4DVAR Cost Function Python}
\begin{lstlisting}[language=Python]
# Import necessary libraries
import numpy as np
import pandas as pd

# Step 1: Load Background State and Observations
# Replace 'background_state.csv' and 'observations.csv' with your actual file paths
background_state = pd.read_csv('background_state.csv').values  # Shape: (N,)
observations = pd.read_csv('observations.csv').values  # Shape: (L, P)

# Load covariance matrices
Q0 = np.load('Q0.npy')  # Shape: (N, N)
R = np.load('R.npy')    # Shape: (PL, PL)

# Step 2: Define Functions
def compute_observation_departure(y_obs, x_forecast, H, delta_x):
    """Calculate the observation departure."""
    return y_obs - H @ (x_forecast + delta_x)

def compute_cost_function(delta_x, Q0_inv, d, R_inv):
    """Calculate the cost function."""
    return delta_x.T @ Q0_inv @ delta_x + d.T @ R_inv @ d

# Step 3: Compute Observation Departures
L = observations.shape[0]
P = observations.shape[1]
N = background_state.shape[0]

# Define H (observation operator)
H = np.random.rand(P, N)  # Replace with your actual observation operator
delta_x = np.zeros(N)  # Initial analysis increment

d = []
for k in range(L):
    y_obs = observations[k]
    departure = compute_observation_departure(y_obs, background_state, H, delta_x)
    d.append(departure)

d = np.concatenate(d)  # Shape: (PL,)

# Step 4: Set Up Covariance Matrices
Q0_inv = np.linalg.inv(Q0)
R_inv = np.linalg.inv(R)

# Step 5: Compute the Cost Function
cost = compute_cost_function(delta_x, Q0_inv, d, R_inv)
print(f'Initial Cost Function: {cost}')

# Step 6: Prepare Data for Quantum Annealing
def prepare_for_quantum(delta_x, alpha, NZ):
    """Prepare data for quantum annealing."""
    G = np.zeros((N, NZ))
    for i in range(N):
        G[i, :] = [-2**(NZ-1-j) for j in range(NZ)]  # Example mapping logic

    b = np.random.randint(0, 2, size=(NZ,))  # Example binary vector
    return G, b

# Example parameters
alpha = 1.0
NZ = 10  # Number of qubits

# Prepare G and b
G, b = prepare_for_quantum(delta_x, alpha, NZ)

# Step 7: Reformulate the Cost Function for Quantum Annealing
A = (1 / alpha**2) * (G.T @ (Q0_inv + ...))  # Fill in remaining terms
u = (-2 / alpha) * (d.T @ R_inv @ H @ G)

# Output the matrices for quantum annealer
print('Matrix A:', A)
print('Vector u:', u)

\end{lstlisting}


\end{document}

